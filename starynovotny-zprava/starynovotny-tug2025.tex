\RequirePackage{luatex85}
\PassOptionsToPackage{shorthands=off}{babel}
\makeatletter
\disable@package@load{fontenc}
\makeatother
\let\oldlooseness=\looseness
\documentclass{csbulletin}
\selectlanguage{czech}
\usepackage{afterpage}
\usepackage{titlesec}
\titlelabel{\thetitle\enspace}
\makeatletter
\ExplSyntaxOn
\cs_new_protected:Npn
  \sanskrit
  {
    \tl_set:Nn
      \l_tmpa_tl
      { 0.9 * }
    \tl_put_right:NV
      \l_tmpa_tl
      \f@size
    \fp_set:NV
      \l_tmpa_fp
      \l_tmpa_tl
    \font\tmpfont={"Lohit Devanagari":mode=harf;script=Devanagari}~at~\fp_to_dim:N \l_tmpa_fp
    \tmpfont
  }
\ExplSyntaxOff
\makeatother
\font\linguistix="NewCMUncial10-Book.otf"
\usepackage{tikz}
\def\slant#1#2{%
  \tikz[baseline=(X.base), xslant=tan(#1)]
    \node[inner sep=0pt, xslant=tan(#1)](X){#2};%
}
\usepackage{luavlna}
\usepackage[strict]{lua-widow-control}
\usepackage{csquotes}
\usepackage[
  backend=biber,
  citestyle=numeric-comp,
  bibstyle=iso-numeric,
  sortlocale=cs,
  autolang=other,
  bibencoding=UTF8,
  mincitenames=2,
  maxcitenames=2,
  doi=true,
  isbn=true,
  shortnumeration=true,
]{biblatex}
\renewcommand\multicitedelim{\addsemicolon\space}
\addbibresource{starynovotny-tug2025.bib}
\usepackage[
  implicit=false,
  hidelinks,
]{hyperref}
\usepackage{hologo}

\ExplSyntaxOn
\newcommand
  \acro
  [ 1 ]
  {
    \tl_set:Nn
      \l_tmpa_tl
      { #1 }
    \regex_replace_all:nnN
      { [^\d]+ }
      { \c{textsc} \cB\{ \c{MakeLowercase} \cB\{ \0 \cE\} \cE\} }
      \l_tmpa_tl
    \regex_replace_all:nnN
      { \d+ }
      { \c{oldstylenums} \cB\{ \0 \cE\} }
      \l_tmpa_tl
    \tl_use:N
      \l_tmpa_tl
  }
\ExplSyntaxOff
\newcommand\pkg{\textsf}
\makeatletter
\DeclareRobustCommand{\La}{L\kern-.36em%
  {\sbox\z@ T%
   \vbox to\ht\z@{\hbox{\check@mathfonts
                        \fontsize\sf@size\z@
                        \math@fontsfalse\selectfont
                        A}%
                  \vss}%
  }}
\DeclareRobustCommand{\Ka}{K\kern-.16em%
  {\sbox\z@ T%
   \vbox to\ht\z@{\hbox{\check@mathfonts
                        \fontsize\sf@size\z@
                        \math@fontsfalse\selectfont
                        A}%
                  \vss}%
   \kern-.12em%
  }}
\makeatother
\def\AllTeX{(\La\kern-.075em)\kern-.075em\TeX}
\def\strankasclankem#1#2{\textbf{??}}
\def\XDP{X\kern-0.25pt\raisebox{-2pt}{D}\kern-0.25pt P}

\begin{document}

\singlechars{czech}{AaIiVvOoUuSsZzKk}

\title{Zpráva z konference TUG 2025}
\EnglishTitle{Report from the TUG 2025 Conference}
\author{Vít Starý Novotný}
\podpis{Vít Starý Novotný, witiko@mail.muni.cz}
\maketitle

\vspace{-1.5em}

Od pátku 18. do neděle 20.\,července se v indické Kérale po 14 letech\footnote{Pátral jsem v \href{https://dml.cz/handle/10338.dmlcz/148724}{archivu Zpravodaje} po zmínkách o \acro{TUG}u \acro{2011}, který se rovněž konal v Kérale. Mé pátrání ale nebylo korunováno úspěchem, nejspíš i kvůli překryvu s dvoukonferencí \TeX perience a Stakan~\cite{striz2011kniha, striz2011ucime}, kterou \CSTUG{} v témže roce spolupořádal v Železné rudě. Navzdory chybějícím publikacím ve Zpravodaji se ale konference zúčastnili Petr Sojka, který zde hovořil o vyhledávání matematiky~\cite{sojka2011why}, a Karel Skoupý, který ve dvou příspěvcích hovořil o implementaci datových struktur pomocí primitivů \hologo{eTeX}u~\cite{skoupy2021data} a o přípravě kapesních slovníků v \TeX u~\cite{skoupy2021typesetting}.} konala konference \acro{TUG 2025}. Autor přítomného textu se konference aktivně zúčastnil a ve zbytku článku referuje o průběhu konference.

\medskip

Ve čtvrtek ještě před oficiálním zahájením konference proběhl již tradičně workshop věnovaný tvorbě přístupných \acro{PDF} dokumentů podle standardu \acro{PDF/UA} v \LaTeX u. Workshop tentokrát proběhl v kampusu firmy \href{https://stmdocs.in/}{\acro{STM Docs}} -- generálního sponzora a organizátora konference. Po návratu z kampusu probíhala v konferenčním hotelu Hyatt Regency uvítací recepce.\footnote{Programu recepce se autor článku nezúčastnil a místo toho strávil den průzkumem města~\cite{starynovotny2025static}.}

\medskip

Konferenci oficiálně započal v pátek ráno Rob Schrauwen z dánského nakladatelství Elsevier, který ve své ,,keynote`` přednášce~\autocites{schrauwen2025true}[čas 13:00]{youtube2025day1a} hovořil o nedostatcích současných postupů pro značkování vědeckých dokumentů. V následující přednášce~\autocites{braun2025state}[čas 1:15:18]{youtube2025day1a} hovořil Erik Braun z německého spolku \acro{DANTE} o současném stavu webového archivu \TeX ového softwaru \acro{CTAN}.

Dopolední sekce se věnovala popularizaci \LaTeX u. Nejprve hovořil Vrajarāja Govinda z Mezinárodní společnosti pro vědomí Kršny\footnote{Sanskrtské jméno je {\sanskrit कृष्ण}, což se vyslovuje Kršna: Souhlásky [š] i [n] jsou retroflexní, takže se při správné sanskrtské výslovnosti vysloví opřením špičky jazyka o střed horního patra. Při přepisu do češtiny se podle českých indologů u jmen bohů vychází ze sanskrtu. (Pozn. red.)} (\acro{ISKCON}) o využití \Hologo{XeLaTeX}u pro přípravu dokumentů ve svém ášramu~\autocites{govinda2025howa,govinda2025howb}[čas 18:48]{youtube2025day1b}. Následně přednášel Jean-Michel Hufflen z francouzského spolku \acro{GUT}enberg o svých zkušenostech z vysokoškolské výuky \LaTeX u~\autocites{hufflen2025whata,hufflen2025whatb}[čas 55:47]{youtube2025day1b}.

Po obědě proběhlo skupinové focení u venkovního bazénu, vizte Obrázek~\ref{fig:group-photo}.

\afterpage{\begin{figure}[t]
\centering
\vspace{-4pt}
\setlength{\belowcaptionskip}{-12pt}
\includegraphics[width=\linewidth]{figs/tug25-conf-group-photo.jpg}\\[-0.2cm]
\caption{Skupinová fotografie účastníků~\cite{shaji2025tug}. Přetištěno se svolením vydavatele.}
\label{fig:group-photo}
\end{figure}}

Odpolední sekce se věnovala přípravě přístupných \acro{PDF} dokumentů v \LaTeX u. Nejprve hovořili Frank Mittelbach a Ulrike Fischer z projektu \LaTeX, kteří ve svých přednáškách~\autocites{mittelbach2025how,fischer2025status,fischer2025news}[časy 1:38:35 a 2:23:00]{youtube2025day1b} poreferovali o současném stavu podpory standardu \acro{PDF/UA} v \LaTeX u. Sekci uzavřel Ross Moore z Macquarijské univerzity, který referoval o výzvách spojených s generováním přístupných bibliografií pomocí \LaTeX ového balíčku Bib\LaTeX~\autocites{moore2025tagginga,moore2025taggingb}[čas 2:59:20]{youtube2025day1b}.

Večerní sekce se věnovala použití \TeX u ve větších systémech. Nejprve řečnil CV~Radhakrishnan, zakladatel firem River Valley Technologies a \acro{STM Docs} a spolku \acro{TUG} India, o archivaci malajálamských dokumentů a jejich převodu do výstupních formátů \acro{HTML} a \acro{PDF} skrz formáty \LaTeX{} a \acro{TEI}~\autocites{cv2025digitala,cv2025digitalb}[čas 3:45:52]{youtube2025day1b}. Sekci uzavřela Doris Behrendt z německého spolku \acro{DANTE}, která hovořila o použití \TeX u pro dynamickou sazbu matematiky v digitálním učebním materiálu pro výuku kryptologie pomocí javascriptové knihovny \Ka\TeX~\autocites{behrendt2025latexa,behrendt2025latexb}[čas 4:25:28]{youtube2025day1b}.

Páteční program uzavřela dobrovolná noční prohlídka města autobusem.

\medskip

Sobotní ranní sekce se věnovala statické analýze \TeX u. Nejprve pojednával KV Rajeesh z \href{https://rachana.org.in/index.html}{Typografického institutu Rachana (\acro{RIT})}\footnote{Rajeesh publikoval v časopisu \emph{TUGboat}~\cite{rajeesh2025tuga} a na svém blogu~\cite{rajeesh2025tugb} vlastní zprávu z konference.} o přípravě OpenType fontu s vestavěným zvýrazňováním syntaxe \TeX u~\autocites{rajeesh2025opentypea,rajeesh2025opentypeb}[čas 11:27]{youtube2025day2}. Následně referoval autor článku o statické analýze programovacího jazyka expl3~\autocites{starynovotny2025expltoolsa,starynovotny2025expltoolsb}[čas 50:23]{youtube2025day2},\footnote{První verze této přednášky zazněla na loňském valném shromáždění \CSTUG u~\cite{starynovotny2024valna} a následně při příležitosti návštěvy Franka Mittelbacha v Brně~\cite{starynovotny2024frank}.} vizte Obrázek~\ref{fig:expltools}.

\afterpage{\begin{figure}[t]
\setlength{\belowcaptionskip}{-12pt}
\hspace{16mm}\includegraphics[width=0.618\linewidth]{figs/vitek-behem-prednasky.jpg}\\[-0.85cm]
\includegraphics[width=\linewidth]{figs/d2-t11-starynovotny-expltools-slides.pdf}\\[-1.21cm]
\caption{Autor článku přednáší o statické analýze programů v jazyce expl3.}
\label{fig:expltools}
\end{figure}}

Dopolední sekce byla vyhrazena pro příspěvky, které tematicky nezapadaly do žádné jiné sekce. Nejprve hovořili Saravanan Murugaiah a Mukesh Subramaniam z firmy C\&M Digitals o komerčním nástroji em\acro{DOX} pro konverzi \LaTeX ových dokumentů do formátu \acro{OOXML}~\autocites{saravanan2025typographic}[čas 1:36:08]{youtube2025day2}. Následně řečnil Salil B. o svém nápadu automaticky zaopatřit záhlaví tabulek symboly, které udávají seřazenost hodnot v jednotlivých sloupcích~\autocites{salilb2025constructing}[čas 2:15:45]{youtube2025day2}. To má usnadnit explorativní analýzu dat a upozornit autora i čtenáře na možné korelace v datech.

Odpolední sekce se týkala lingvistiky. Nejprve SK Venkatesan\footnote{Česká norma přepisu a zápisu indických jmen je rozkolísaná. Proto u jmen účastníků používám formu, která je uvedena v konferenčním programu~\cite{tug2025program}.} z firmy \acro{TNQ} Tech porovnal vývoj alfabetických a logografických/slabičných písem na příkladech protoelamštiny a harappského písma~\autocites{venkates2025analphabetica}{venkates2025analphabeticb}[čas 3:02:14]{youtube2025day2}. Následně pojednávala Vaishnavi Murthy Yerkadithaya z firmy Monotype o standardizaci indického písma tigalari,\footnote{Písmo je součástí verze 17 standardu Unicode~\cite[Sekce~15.18]{unicode17}.} které se používá pro zápis tuluštiny, kannadštiny a sanskrtu~\autocites{yerkadithaya2025encoding}[čas 3:38:52]{youtube2025day2}. Sekci uzavřel {\sanskrit निरंजन} (Niranjan) přednáškou o \LaTeX ovém balíčku {\linguistix LinguisTiX} pro sazbu lingvistických textů~\autocites{niranjan2025linguisticsa}{niranjan2025linguisticsb}{venkates2025analphabeticb}[čas 4:14:40]{youtube2025day2}.\footnote{Projekt autora článku i {\linguistix LinguisTiX} finančně podpořil rozvojový fond \acro{TUG}u~\cite{tug2025grants}.}

Večerní sekce byla o bibliografii. Nejprve Norman Gray z Glasgowské univerzity představil program Beastie, jenž může nahradit Bib\TeX{} a zároveň nabízí knihovnu funkcí v jazyce Scheme pro práci s \texttt{.bib} databázemi~\autocites{gray2025beastiea}{gray2025beastieb}{venkates2025analphabeticb}[čas 5:06:22]{youtube2025day2}. Následně referoval Martin J. Osborne z Torontské univerzity o nástroji \href{https://text2bib.org/}{\texttt{text2bib}} pro převod textových referencí do formátu \texttt{.bib}~\cite[čas 5:38:28]{youtube2025day2}.
\oldlooseness=-1

Sobotní program uzavřel sitárový koncert a taneční představení Kathakali.

\medskip

Nedělní ranní sekce se, podobně jako pondělní večerní, věnovala rozsáhlejším systémům poháněným \TeX em. Nejprve hovořil T Rishikesan Nair z~firmy \acro{STM~DOCS} o komerčním nástroji Primo pro přípravu, korektury a redakční úpravy časopiseckých článků~\autocites{nair2025efficientlya}[čas 00:13]{youtube2025day3}. Sekci uzavřel Erik Nijenhuis ze společnosti Xerdi, který hovořil o svém systému \XDP{} pro sazbu smluv pomocí \LaTeX u, Markdownu a \acro{YAML}u~\autocites{nijenhuis2025xerdia}{nijenhuis2025xerdib}[čas 35:40]{youtube2025day3}.

Zbylé nedělní sekce neměly vyhraněná témata. V dopolední sekci podruhé vystoupila Doris Behrendt, která tentokrát představila každoroční akci Chaos Communication Camp pořádanou německým spolkem Chaos Computer Club (\acro{CCC}). Akce zahrnuje sekci \LaTeX{} Village zaměřenou na \TeX{} a související technologie~\autocites{behrendt2025latexc}[čas 1:08:10]{youtube2025day3}. Následně řečnila Veeraraghavan Balasubramanian z nakladatelství Elsevier o výzvách, kterým dnes čelí nakladatelé~\autocites{balasubramanian2025challengesa}{balasubramanian2025challengesb}[čas 1:50:12]{youtube2025day3}, čímž tematicky navázala na pondělní přednášku Roba Schrauwena.

Po obědě proběhlo valné shromáždění \acro{TUG}u, na kterém bývalý předseda a současný člen výboru Boris Veytsman informoval o aktivitách sdružení, rozhodnutích výboru, stavu financí a vývoji členské základny~\cite{hefferton2025tug}. Mezi hlavní aktivity patřilo vydání \TeX{} Live 2025, podpora vývoje softwaru a financování účasti na konferencích. Výbor mj. udělil doživotní čestné členství Barbaře Beeton a Karlu Berrymu za jejich služby komunitě, zavedl zlevněné zkušební elektronické členství za 20 dolarů na první rok a rozhodl, že \acro{TUG 2026} se uskuteční v kanadské Albertě. Největší výdaje tradičně představuje tisk a distribuce časopisu \emph{TUGboat} a personální náklady. Počet členů dlouhodobě klesá a letos prorazil psychologickou hranici 1\,000 členů.
\oldlooseness=-1

V odpolední sekci vystoupil nejprve Boris Veytsman, který pojednával o svých makrech pro zanášení redakčních poznámek do \LaTeX ových dokumentů~\autocites{veytsman2025systema}{veytsman2025systemb}[čas 2:26:50]{youtube2025day3}. Následně přednášel Rahul Krishnan S o podpoře standardu \acro{PDF/UA} v komerčním nástroji Neptune pro přípravu, korektury a redakční úpravy časopiseckých článků~\autocites{krishnan2025implementinga}{krishnan2025implementingb}[čas 3:01:00]{youtube2025day3}. Sekci uzavřel Mathias Jakobsen ze společnosti Overleaf přednáškou o formální gramatice \LaTeX u, kterou používá \acro{WYSIWYG} mód textového editoru Overleaf~\cite{novotny2021overleaf} pro statickou analýzu struktury \LaTeX ových dokumentů~\autocites{jakobsen2025besta}{jakobsen2025bestb}[čas 3:26:20]{youtube2025day3}.

\medskip

Na závěr konference se uskutečnil workshop kéralského kaligrafa Narayana Bhattathiri, po kterém následoval banket.

% https://tug.org/tug2025/participants.html
% https://tug.org/tug2025/program.html
% https://tug.org/TUGboat/Contents/contents46-2.html

\begingroup
\sloppy
\printbibliography
\endgroup

\begin{summary}
From Thursday, July 17, to Sunday, July 20, the \acro{TUG 2025} conference was held in Kerala, India, for the first time in 14 years. The author of this article took an active part in the conference and reports on its proceedings.
\end{summary}
\end{document}