\RequirePackage{luatex85}
\PassOptionsToPackage{shorthands=off}{babel}
\makeatletter
\disable@package@load{fontenc}
\makeatother
\let\oldlooseness=\looseness
\documentclass{csbulletin}
\selectlanguage{czech}
% TODO: Odstranit po přepisu těla tak, aby nepoužívalo `frame`, `mode`, apod.
% \usepackage{beamerarticle}
\usepackage{csquotes}
\usepackage{luavlna}
\usepackage{booktabs}
\usepackage{fancyvrb}
\newcommand{\meta}[1]{\ensuremath{\langle}\textrm{\textit{#1}}\ensuremath{\rangle}}
% \usepackage{subfigure}
\usepackage{subcaption}
\usepackage{linebreaker}
\usepackage[strict]{lua-widow-control}
\setcounter{secnumdepth}{3}
% TODO: Odkomentovat poté, co se zbavíme balíčku `beamerarticle`.
%\usepackage{titlesec}
%\titlelabel{\thetitle\enspace}
\usepackage[
  backend=biber,
  citestyle=numeric-comp,
  bibstyle=iso-numeric,
  sortlocale=cs,
  autolang=other,
  bibencoding=UTF8,
  mincitenames=2,
  maxcitenames=2,
  doi=true,
  isbn=true,
  shortnumeration=true,
]{biblatex}
\renewcommand\multicitedelim{\addsemicolon\space}
\addbibresource{hoftich-article.bib}
\usepackage[
  implicit=false, % tato volba vede k chybě v příkazu \cite, je skutečně třeba?
  % VSN: Je třeba kvůli finálnímu Zpravodaji, kde by se jinak rozbil makry hyperrefu soubor .toc, ze kterého automaticky generujeme řádek v obsahu Zpravodaje. Chyba v příkazu \cite by snad měla být jen dočasná; klidně si teď pro své úpravy "implicit=false" vypni a já to vyřeším před tiskem.
  hidelinks,
]{hyperref}
\ExplSyntaxOn
\newcommand
  \acro
  [ 1 ]
  {
    \tl_set:Nn
      \l_tmpa_tl
      { #1 }
    \regex_replace_all:nnN
      { [^\d]+ }
      { \c{textsc} \cB\{ \c{MakeLowercase} \cB\{ \0 \cE\} \cE\} }
      \l_tmpa_tl
    \regex_replace_all:nnN
      { \d+ }
      { \c{oldstylenums} \cB\{ \0 \cE\} }
      \l_tmpa_tl
    \tl_use:N
      \l_tmpa_tl
  }
\ExplSyntaxOff
\begin{document}
\singlechars{czech}{AaIiVvOoUuSsZzKk}
\hyphenation{GitHub}

\title{Publikace \LaTeX ových dokumentů na webu pomocí \TeX 4ht a GitHub Actions}
\EnglishTitle{Publishing \LaTeX\ Documents on the Web Using \TeX 4ht and GitHub Actions}
\author{Michal Hoftich}
\podpis{Michal Hoftich, michal.h21@gmail.com}
\maketitle

\begin{abstract}
Článek představí sadu šablon pro nástroj \TeX 4ht, který slouží k převodu
\LaTeX ových dokumentů do \acro{HTML}. Tyto šablony výrazně usnadňují publikaci různých
typů dokumentů na webu a přinášejí možnosti zpracování a automatizace.

První šablona je určena pro převod knižních dokumentů do webové podoby.
Umožňuje rozdělení textu do jednotlivých kapitol s automaticky generovanou
navigací a podporou responzivního designu, takže je výsledek dobře čitelný i na
mobilních zařízeních.

Druhá šablona slouží k tvorbě staticky generovaných blogů. Každý příspěvek je
psán jako samostatný \LaTeX ový dokument, který je pomocí \TeX 4ht převeden do
\acro{HTML}. Následně jsou tyto články zpracovány statickým generátorem webů, jako je
například Jekyll, který se postará o sestavení celého blogu, vytvoření
rozcestníků, archivů a další navigace.

Třetí šablona je zaměřena na převod prezentací vytvořených v prostředí Beamer
do formy tzv. handoutů -- přehledových materiálů pro posluchače. Výsledkem je
čitelný a dobře strukturovaný webový dokument vhodný pro sdílení po přednášce.

Všechny šablony jsou navrženy tak, aby fungovaly v rámci GitHub Actions. To
znamená, že dokumenty mohou být automaticky zkompilovány a publikovány online
pokaždé, když dojde ke změně v repozitáři. 
\end{abstract}
\klicovaslova: \TeX 4ht, \LaTeX, \acro{HTML}, Jekyll, Beamer, GitHub Actions



\section{Úvod do \TeX4ht}

\TeX4ht je nástroj pro konverzi z \LaTeX u do \acro{HTML} a dalších formátů, jako jsou
\acro{ODT}, \acro{EPUB} nebo \acro{JATS XML}. Dokáže zachovat strukturu i
formátování původního dokumentu a nabízí různé metody pro matematický výstup,
včetně obrázků, MathML a MathJaxu. Umožňuje také generovat obrázky přímo z
výstupu \LaTeX u a podporuje běžně používané \LaTeX ové balíčky.

\TeX4ht samotný je především balíček, který redefinuje příkazy jiných balíčků tak, aby
vkládaly tagy výstupních formátů.
% Vždy se kompiluje do DVI výstupu, který se poté zpracovává
% dalšími nástroji, které vytvoří soubory ve výstupním formátu, \acro{CSS} soubor, nebo obrázky.
Části \acro{DVI} výstupu lze konvertovat na obrázky ve formátu \acro{PNG} nebo \acro{SVG}. Toho se využívá pro podporu
obrázků tvořených pomocí balíčků Ti$k$Z nebo PSTricks.

Více informací o \TeX4ht a jeho základním použití naleznete v mém starším článku ve \emph{Zpravodaji}~\autocite{hoftich2018-tex4ht}. 
V tomto článku se zaměřím na pokročilejší možnosti využití \TeX4ht a představení šablon pro publikaci dokumentů na webu s využitím GitHub Actions.


% \subsection{GitHub Actions}
\section{Automatické generování HTML výstupu pomocí GitHub Actions}

\subsection{Základní struktura konfigurace GitHub Actions}

GitHub, ale i další repozitáře jako GitLab nebo Bitbucket,  umožňuje
spouštět definované scénáře (např. build, testy nebo nasazení) v reakci na
události v repozitáři (push, pull request apod.).

Tato funkce se nazývá GitHub Actions~\cite{github-actions-quickstart}. V tomto článku si ukážeme, jak ji využít pro automatizovanou
kompilaci dokumentů pomocí \TeX4ht a jejich publikování na webu.

Šablony pro \TeX4ht, které dále představím, jsou navrženy tak, aby fungovaly v rámci GitHub Actions. To
znamená, že dokumenty mohou být automaticky zkompilovány a publikovány online
pokaždé, když dojde ke změně v repozitáři. Tento přístup zajišťuje, že je
webová verze dokumentu vždy aktuální.



Sada pravidel, která definují, jaké akce se mají provést v návaznosti na nějakou událost v repozitáři, se v terminologii
GitHub Actions nazývá \textit{workflow}.
Workflow je definován v souboru ve formátu \acro{YAML}, který je po stažení šablon dostupný v podadresáři \texttt{.github/workflows/main.yml}.
\acro{YAML} je jednoduchý textový formát používaný pro konfigurace. GitHub Actions v něm definují jednotlivé kroky automatizace.
Například v šabloně pro \uv{handouty} prezentací se používá následující kód:


% \begin{frame}[fragile]{Přehled GitHub Actions}

  % \begin{block}{Klíčové části workflow pro sestavení a publikování HTML:}

\begin{verbatim}
  - name: Spuštění make4ht
    uses: xu-cheng/texlive-action/full@v1
    with:
      run: |
        make4ht -lj index -a debug -d out handout.tex

  - name: Publikování webových stránek
    uses: peaceiris/actions-gh-pages@v3
    with:
      github_token: ${{ secrets.GITHUB_TOKEN }}
      publish_dir: ./out
\end{verbatim}
% \end{block}

% \end{frame}

Jednotlivé kroky úlohy jsou definovány v sekci \texttt{steps}, jejíž obsah zobrazuje předešlý úryvek kódu. 
% Každý krok typicky spouští Docker kontejner, který má k dispozici soubory z repozitáře a také soubory vytvořené v předešlých krocích.
Každý krok provede určitou akci, například kompilaci \LaTeX u. K dispozici má soubory z repozitáře a také soubory vytvořené v předešlých krocích.
V této ukázce se nachází dva kroky, které 
používají dvě GitHub Actions -- \texttt{xu-cheng/\-texlive-action}~\cite{xu-cheng:texlive-action}
a \texttt{peaceiris\-/actions-\-gh-pages}~\cite{peaceiris:actions-gh-pages}.
První z nich umožňuje používat libovolný příkaz dostupný v instalaci \TeX\ Live jako \texttt{make4ht} nebo \texttt{lualatex}.
Druhá publikuje obsah zadaného adresáře na webu pomocí GitHub Pages.


Pro spuštění akce je třeba pouze nahrát změny do repozitáře pomocí následujících příkazů:

\begin{Verbatim}[commandchars=\|()]
$ git add |meta(upravené soubory)
$ git commit -m "Popis změn"
$ git push origin main
\end{Verbatim}

Příkaz \texttt{git add} přidá upravené soubory do indexu pro commit, \texttt{git commit} vytvoří nový commit s popisem změn
a \texttt{git push} nahraje změny do větve \texttt{main} na vzdáleném repozitáři.
Při každé změně v repozitáři se poté spustí workflow,
který zkompiluje soubor \texttt{handout.tex} do \acro{HTML} pomocí příkazu 
\texttt{make4ht}:
\begin{verbatim}
$ make4ht -lj index -a debug -d out handout.tex
\end{verbatim}

Tento příkaz vytvoří \acro{HTML} soubory v adresáři \texttt{out} pomocí volby \texttt{-d out}. 
Volba \texttt{-lj index} je zkratkou pro volby \texttt{--lua} která vyvolá kompilaci Lua\TeX em a \texttt{--jobname index}, která  nastaví název výstupního \acro{HTML} souboru na \texttt{index.html}.
Akce \texttt{peaceiris\-/actions-gh-pages} poté zveřejní všechny soubory v adresáři 
\texttt{out}, což je  specifikované nastavením vlastnosti \texttt{publish\_dir} ve workflow souboru.

\acro{HTML} verze prezentace bude dostupná na následující adrese:

\begin{Verbatim}[commandchars=\|()]
https://|meta(jméno uživatele).github.io/|meta(název repozitáře)/
\end{Verbatim}

V \acro{URL} není třeba specifikovat název souboru -- GitHub Pages
automaticky nabízí soubor \texttt{index.html}. Tato funkce usnadňuje sdílení
prezentace a pomáhá předejít nefunkčním odkazům kvůli neshodě názvů souborů.

\subsection{Sledování stavu workflow a řešení chyb}


\begin{figure}
  \centering
\includegraphics[width=\textwidth]{img/github-actions-new.png}
\caption{Rozhraní GitHub Actions zobrazující stav workflow}
\label{fig:github-actions}
\end{figure}

Po každém nahrání změn  můžete zkontrolovat záložku
\texttt{Actions} ve svém GitHub repozitáři (vizte Obrázek~\ref{fig:github-actions}). 
Zobrazuje stav \textit{workflow}, včetně
informace o úspěšném dokončení nebo případných chybách.

\begin{figure}
  \centering
\includegraphics[width=\textwidth]{img/github-error-new.png}
\caption{Příklad chyby v GitHub Actions workflow}
\label{fig:github-error}
\end{figure}

Můžete také zkontrolovat logy běhu workflow, abyste viděli, co se pokazilo.
Pokud narazíte na chybu, bude zobrazena v logu a tyto informace můžete
použít k řešení problému.

Například v logu na Obrázku~\ref{fig:github-error} vidíme, že u kroku 
\texttt{Run make4ht} se zobrazila červená výstražná ikonka. Po kliknutí 
na název kroku se zobrazí podrobnosti o chybě.
V tomto případě byl nesprávný název hlavního \TeX ového souboru, který se 
kompiloval pomocí \texttt{make4ht}, což nám oznamuje hláška \enquote{\texttt{make4ht: missing required parameter: filename}}.
V \acro{YAML} souboru GitHub Actions tedy musíme opravit název souboru na správný.

\subsection{Publikování HTML verze}

\begin{figure}
  \centering
\includegraphics[width=\textwidth]{img/github-pages-new.png}
\caption{Nastavení GitHub Pages pro repozitář}
\label{fig:github-pages}
\end{figure}

Po úspěšném dokončení workflow můžete nastavit GitHub Pages pro zobrazování obsahu větve \texttt{gh-pages}.
Výstupní soubory vytvořené pomocí \texttt{make4ht} budou dostupné na adrese
\verb|https://|\meta{jméno uživatele}\verb|.github.io/|\meta{název repozitáře}\verb|/|.

Aby se \acro{HTML} verze dokumentu zobrazila správně, je třeba v nastavení GitHub repozitáře
nastavit GitHub Pages tak, aby používala větev \texttt{gh-pages} jako zdroj pro publikování (vizte Obrázek~\ref{fig:github-pages}).
Tuto větev používá akce \texttt{peaceiris/actions-gh-pages} pro publikování výstupních souborů.
Díky tomu, že se používá samostatná větev, zůstává hlavní větev \texttt{main} čistá a obsahuje pouze zdrojové \TeX ové soubory.


\subsection{Využití šablon na GitHubu}
\begin{figure}
  \centering
  \includegraphics[width=\textwidth,alt={}]{img/template-use-new.png}
  \caption{Tlačítko pro použití šablony na GitHubu}
  \label{fig:template-use}
\end{figure}

Pro využití šablon na GitHubu klikněte na tlačítko
\enquote{Use this template} na stránce GitHub repozitáře s podporou šablon (vizte Obrázek~\ref{fig:template-use}). 
Po vyplnění dialogu pro vytvoření nového repozitáře z šablony se ve vašem účtu vytvoří
nový repozitář se stejnou strukturou a se stejnými soubory, jako měl původní repozitář, ovšem bez jeho 
Git historie. Poté nově vytvořený repozitář můžete naklonovat na svůj počítač a začít upravovat 
obsah souborů. 

\begin{figure}
  \centering
  \includegraphics[width=\textwidth,alt={Dialog vytvoření nového repozitáře ze šablony}]{img/new-repo-new.png}
  \caption{Dialog vytvoření nového repozitáře ze šablony}
  \label{fig:new-repo}
\end{figure}


\section{Šablona pro webové verze knih}

Tato šablona~\cite{hoftich:tex4ht-booksite} je navržena tak, aby umožňovala export každé kapitoly do
samostatné \acro{HTML} stránky. Je ideální pro publikaci knihy, dokumentace nebo
skript, kde každá kapitola tvoří vlastní webovou stránku propojenou navigací.


Pokud chceme vytvořit samostatné \acro{HTML} soubory pro jednotlivé kapitoly našeho dokumentu,
můžeme použít následující příkaz:
\begin{verbatim}
$ make4ht document.tex "2"
\end{verbatim}


Číselné volby umožňují nastavit, od které úrovně členění textu se mají vytvářet
samostatné soubory. Například volba s hodnotou 1 způsobí, že budou vytvořeny
samostatné soubory pro nejvyšší úroveň struktury dokumentu. Hodnota 2 zajistí
rozdělení i pro další úroveň, obvykle kapitoly, případně sekce, pokud typ
dokumentu kapitoly neobsahuje. Při volbě 3 se soubory vytvářejí i pro sekce, a
to i u typů dokumentů, které běžně kapitoly obsahují. Takto lze pokračovat až
do úrovně 7, kdy jsou samostatné soubory generovány i pro velmi jemné členění,
na úrovní příkazů \verb|\paragraph|.
Obecně doporučuji použít volbu \verb|"2"|.



\subsection{Pojmenování souborů kapitol}

Ve výchozí podobě jednotlivé soubory kapitol obsahují jen základní navigaci
s odkazy na předchozí a následující kapitolu a na hlavní soubor.
Názvy souborů jsou tvořeny názvem hlavního souboru, následovaným zkratkou úrovně nadpisu a pořadím dané úrovně nadpisu v dokumentu.
Například pro dokument s názvem \texttt{document.tex} budou názvy souborů pro jednotlivé kapitoly \texttt{documentch1.html}, \texttt{documentch2.html} 
a soubor s obsahem, titulní stránkou a textem, který se nenachází uvnitř žádné kapitoly, bude pojmenovaný \texttt{document.html}.
Pro nečíslované kapitoly, například rejstřík nebo seznam literatury, se vytváří soubor s koncovkou \texttt{li}, například 
\texttt{documentli1.html}.

\begin{table}
  \centering
  \begin{tabular}{llll}
    \toprule
    Název kapitoly & výchozí název & sec-filename & sec-slugname \\
    \midrule
    S háčky & \texttt{documentch1.html} & \texttt{Sháčky.html} & \verb|s_hacky.html| \\
    \bottomrule
  \end{tabular}
  \caption{Názvy souborů kapitol podle různých způsobů pojmenování}
  \label{tab:chapter-filenames}
\end{table}


Problém nastane, pokud změníme pořadí kapitol nebo přidáme kapitolu novou. Odkazy z externích webů na sekce jednotlivých kapitol mohou přestat fungovat, 
protože se jejich obsah přesune do jiného souboru. Abychom tomu předešli, můžeme využít jiné způsoby pojmenování souborů.
\TeX4ht nabízí dvě možnosti, jak pojmenovat soubory na základě názvů kapitol. První z nich je \texttt{sec-filename}, druhá
\texttt{sec-slugname}. Jejich použití ukazuje Tabulka~\ref{tab:chapter-filenames}. Z hlediska použitelnosti na webu je lepší volba \texttt{sec-slugname}.
Pro svou funkcionalitu ovšem vyžaduje přepínač \verb|-l|, protože název se vytváří
pomocí jazyka Lua. Volby pro \texttt{make4ht} budou tedy následující:

\begin{verbatim}
$ make4ht -l document.tex "2,sec-slugname"
\end{verbatim}

% Mějme například kapitolu s názvem \textit{Kapitola s háčky}.
% Volba \texttt{sec-filename} vytvoří název souboru na základě názvu kapitoly, včetně diakritiky a velkých písmen, v tomto případě tedy \texttt{Kapitolasháčky.html}.
% Volba \texttt{sec-slugname} vytvoří název souboru, který je lépe použitelný na webu, protože odstraňuje diakritiku a speciální znaky z názvu, velká písmena 
% převádí na malá a mezery nahradí podtržítky. Výsledný název souboru bude tedy bude
% například \texttt{kapitola\_s\_hacky.html}.



\subsection{Boční menu s obsahem}

Problémem aktuálně vytvořených \acro{HTML} stránek je, že neobsahují navigační menu s odkazy na 
všechny kapitoly. Navigace obsahuje pouze odkazy na předchozí a následující kapitolu a 
kořenový soubor. Pro snadnou navigaci mezi kapitolami můžeme využít volbu \texttt{fulltoc}.
Ta vytvoří na každé stránce boční menu obsahující plný obsah knihy (vizte Obrázek~\ref{fig:fulltoc-basic}).
Díky tomu lze snadno přecházet mezi jednotlivými kapitolami. 


Ovšem toto menu může být poměrně rozsáhlé, pokud je v knize hodně podkapitol. Můžeme chtít 
zobrazit pouze sekce aktuální kapitoly a sekce ostatních kapitol skrýt (vizte Obrázek~\ref{fig:collapsetoc}). Toho je možné docílit 
pomocí JavaScriptu nebo DOM filtru \texttt{collapsetoc} v \texttt{make4ht}, což je možnost, kterou šablona využívá.



\begin{figure}
\begin{subfigure}{\textwidth}
\centering
\includegraphics[width=0.95\textwidth]{img/fulltoc-basic-new.png}
\caption{Ukázka bočního menu s obsahem dokumentu}
\medskip
\label{fig:fulltoc-basic}
\end{subfigure}
\begin{subfigure}{\textwidth}
\centering
\includegraphics[width=0.95\textwidth]{img/collapsetoc-new.png}
\caption{Ukázka bočního menu s rozbalenými sekcemi pouze aktuální kapitoly}
\label{fig:collapsetoc}
\end{subfigure}
\caption{Porovnání bočního menu s plným obsahem dokumentu a menu s rozbalenými sekcemi pouze aktuální kapitoly}
\end{figure}




Skrytí menu je užitečné především na mobilních zařízeních s menším displejem, kde by při 
zobrazeném menu nezůstalo dostatek místa pro zobrazení samotného textu dokumentu.
Pro zobrazení menu využíváme JavaScript, kterému se jinak snažíme vyhýbat. 

\subsection{Konfigurační soubor}

Většina barev v šabloně je nastavená pomocí \acro{CSS} proměnných, které můžeme změnit v konfiguračním souboru \texttt{config.cfg}.
K dispozici jsou následující vlastnosti, které můžeme upravit:

    \begin{description}
      \item[maintoc] -- menu s obsahem dokumentu,
      \item[hamburger] -- tlačítko pro skryté menu,
      \item[header] -- hlavička dokumentu,
      \item[footer] -- patička dokumentu.
    \end{description}

    Pro každou z těchto vlastností můžete nastavit barvu popředí a pozadí přidáním \texttt{-color}, respektive \texttt{-background} 
    k jejich názvu.

Například pro změnu barev bočního menu můžeme použít následující kód:

\begin{verbatim}
\Css{
  body{
    --maintoc-background: \#f0f0f0;
    --maintoc-color: \#00AFA0;
  }
}
\end{verbatim}

Příkaz \verb|\Css| slouží pro vložení kódu do \acro{CSS} souboru, který \TeX4ht vytváří. 
Můžete jej vložit do konfiguračního souboru \texttt{config.cfg} 
kamkoliv mezi příkazy \verb|\Preamble| a \verb|\EndPreamble|. Při použití hexadecimálního 
zápisu je třeba escapovat znak \verb|#| pomocí zpětného lomítka, jinak dojde ke
kompilační chybě.

\section{Šablona pro blogování s \LaTeX em}

Tato šablona~\cite{hoftich:tex4ht-blog} umožňuje vytvářet blog pomocí statických generátorů webů.
Využívá přitom \TeX4ht pro generování \acro{HTML} verze jednotlivých článků
ve formátu, který je kompatibilní s většinou statických generátorů webů,
jež podporují \acro{HTML} soubory jako vstupní formát.

Statické generátory webů představují způsob, jak vytvářet a spravovat
webové stránky bez nutnosti provozovat databázi nebo dynamický backend. 
Obecně fungují tak, že z jednoduchých textových souborů (například ve formátu Markdown nebo \acro{HTML})
vytvoří kompletní webové stránky pomocí šablon a dalších pravidel. 
Vytvářejí také archivy článků, \acro{RSS} kanály a další doprovodné funkce, které jsou běžné na blozích.

Výhodou je, že takový web můžeme umístit na téměř libovolný webový server;
nemusíme řešit, zda podporuje například PHP nebo jiný systém pro dynamické 
zobrazování stránek. Například GitHub nabízí GitHub Pages, kam můžeme 
umístit stránky zdarma. V rámci GitHub Actions také můžeme spouštět generátory
automaticky, nebo můžeme využít vestavěnou podporu pro generátor Jekyll.

V naší šabloně je každý článek vytvořen jako samostatný \TeX{}ový dokument
umístěný v odděleném adresáři. Využíváme skript, který prochází všechny
adresáře s články, dokumenty, které se od minulé kompilace změnily, kompiluje je pomocí \TeX4ht 
a ukládá vygenerované \acro{HTML} soubory do adresáře, který slouží jako vstup pro statický generátor webu.
Tyto soubory jsou automaticky uloženy do zvláštní větve GitHub repozitáře,
takže výstupní \acro{HTML} soubory nemusíme spravovat ručně. Ukládáme je proto, abychom 
nemuseli při každém sestavení blogu znovu kompilovat všechny články, ale pouze ty,
které se od minulé kompilace změnily.


\subsection{Formát vstupních souborů}

Statické generátory webů obvykle vyžadují, aby vstupní soubory obsahovaly blok s metadaty
v následujícím formátu:

\begin{verbatim}
---
title: Titulek dokumentu
---
# Úvod
Text dokumentu
\end{verbatim}

Na začátku souboru je umístěn blok s metadaty dokumentu ve formátu \acro{YAML}. Z obou stran
je obklopený řádky obsahujícími pouze tři spojovníky. Za blokem metadat pak následuje 
text dokumentu.

V tomto příkladě používáme Markdown, ovšem stejný princip 
můžeme použít i pro \acro{HTML} soubory produkované pomocí \TeX4ht.
Místo Markdownu totiž můžeme použít i \acro{HTML}, které získáme z obsahu elementu 
\verb|<body>|.

Příkaz \texttt{make4ht} obsahuje rozšíření \texttt{staticsite}, které umožňuje generovat \acro{HTML} soubory
s \acro{YAML} metadaty z \TeX ových dokumentů. Automaticky například přidává
titulek dokumentu získaný z příkazu \verb|\title| a další metadata. Mějme například 
následující jednoduchý \TeX ový dokument:


\begin{verbatim}
\documentclass{article}
\begin{document}
\title{Hello world test}
\author{Michal}
\maketitle
This is my test post.
\end{document}
\end{verbatim}

Tento dokument můžeme zkompilovat do \acro{HTML} s \acro{YAML} preamble pomocí následujícího příkazu:
\begin{verbatim}
$ make4ht -f html5+staticsite filename.tex
\end{verbatim}

Rozšíření \texttt{staticsite} zapneme tím, že jeho název připojíme znaménkem plus za formát souboru 
ve volbě \texttt{-f}. Soubor bude automaticky pojmenován ve formátu \verb|YYYY-MM-DD-|\meta{název originálu}. 
Vygenerovaný soubor bude vypadat následovně:

\begin{verbatim}
---
meta:
- charset: ’utf-8’
- name: ’viewport’
  content: ’width=device-width,initial-scale=1’
time: 1626619562
updated: 1627244699
styles:
- ’2021-07-18-hello-world.css’
title: ’Hello world test’
---
<!--  l. 7  --><p class=’indent’>   This is my test post.
</p>
\end{verbatim}
% \end{frame}

Soubor obsahuje různá metadata získaná z \acro{HTML} souboru. Jednak se jedná 
o obsah elementů \verb|<meta>|, dále časy vytvoření souboru a poslední aktualizace v číselné formě, 
seznam použitých stylů, titulek atd. Tato metadata můžeme používat v šablonách 
použitého generátoru webů.



\subsection{Struktura šablony pro blog}

Šablona má následující strukturu adresářů, která předpokládá použití 
statického generátoru Jekyll \cite{jekyllrb}:

\begin{verbatim}
blog/
.. texposts/
.... first_post/
...... first_post.tex
.... second_post/
...... second_post.tex
.. docs/
.... _posts/
.... css
\end{verbatim}

Každý příspěvek na blogu má vlastní podadresář v adresáři \texttt{texposts}. 
Vygenerované soubory se poté zkopírují do různých podadresářů v adresáři 
\texttt{docs}, který Jekyll použije jako zdrojový adresář pro sestavení blogu. 
Například \acro{HTML} soubory pro blog se umístí do podadresáře 
\verb|_posts| a kaskádové styly do podadresáře \texttt{css}.

Pro start nového blogu je třeba odstranit všechny podadresáře v adresáři \texttt{texposts}
kromě adresáře \texttt{cookiecutter-post}, který slouží jako šablona pro nové příspěvky.

\begin{verbatim}
$ find texposts -mindepth 1 -maxdepth 1 -type d \
  ! -name cookiecutter-post \
  -exec rm -rf {} +
$ git add -u texposts
$ git commit -m "Clean start for a new blog"
$ git push
\end{verbatim}

Nové příspěvky pak můžeme přidávat ručně, vytvořením podadresáře v adresáři \texttt{texposts}
nebo využít šablonu \texttt{cookiecutter-post} pomocí nástroje Cookiecutter \cite{cookiecutter}:

\begin{verbatim}
$ cd texposts
$ cookiecutter cookiecutter-post
\end{verbatim}

Nástroj se zeptá na název příspěvku, autora a další otázky, poté vytvoří nový adresář se základním \TeX ovým souborem.
Pro změnu výchozích hodnot můžeme upravit soubor \texttt{cookiecutter.json} v adresáři \texttt{cookiecutter-post},
\TeX ovou šablonu pak můžeme upravit v souboru \verb|{{cookiecutter.project_slug}}| ve stejném adresáři.

Po dokonečení úprav nového příspěvku jej můžeme přidat do repozitáře, ovšem je třeba také vytvořit soubor, který obsahuje 
timestamp publikace příspěvku. Tento soubor musí mít název \meta{jméno dokumentu}\texttt{.published} a také je třeba přidat jej do repozitáře. 
Na základě timestampu se vytváří neměnná \acro{URL} příspěvku, neboť Jekyll používá datum v názvu souboru pro generování \acro{URL}.
Soubor můžeme vytvořit například pomocí následujícího příkazu:

\begin{Verbatim}[commandchars=\|()]
$ date +%s > |meta(jméno dokumentu).published
\end{Verbatim}

Další možností je zkompilovat dokument pomocí \texttt{make4ht} s využitím módu \texttt{publish}:

\begin{Verbatim}[commandchars=\|()]
$ make4ht -m publish |meta(jméno dokumentu).tex
\end{Verbatim}

Tento příkazh vytvoří soubor s timestampem publikace automaticky a navíc zkopíruje vygenerované \acro{HTML}
nebo \acro{CSS} soubory do vstupního adresáře pro Jekyll. 

Po přidání nového příspěvku a souboru \texttt{.published} do repozitáře se automaticky spustí GitHub Actions workflow, který aktualizuje blog.


\section{Šablona pro prezentace}

Tato šablona~\cite{hoftich:tex4ht-presentation} je určena pro prezentace, které
potřebují víc než jen samotné snímky. Umožňuje vytvořit nejen prezentaci, ale
také textovou verzi prezentace ve formátu \acro{HTML} (vizte Obrázek~\ref{fig:handout}). 
Šablona automaticky vytváří \acro{HTML} verzi článku, která je poté zveřejněna na
GitHub Pages. Veškeré materiály se generují z jednoho
zdrojového souboru, čímž se zjednodušuje správa celého projektu a zajišťuje
se konzistence mezi jednotlivými výstupy.

\begin{figure}
\centering
\includegraphics[width=\textwidth]{img/handout-new.png}
\caption{Ukázka článku z prezentace v \protect\acro{HTML} formátu}
\label{fig:handout}
\end{figure}

Součástí projektu je několik souborů, které jsou používané pro generování
různých výstupů prezentace.
Soubor \texttt{slides.tex} slouží ke generování hlavní prezentace ve formátu
Beamer a obsahuje vše, co se zobrazuje během přednášky. Pro vytvoření textové,
čtenářsky přívětivé verze prezentace je určen \texttt{handout.tex}, který je
formátován jako klasický článek. Kromě obsahu viditelného na snímcích zahrnuje
i doplňující poznámky a komentáře a je koncipován tak, aby byl samostatně
srozumitelný i pro ty, kteří prezentaci neviděli. Samotný obsah prezentace je
uložen v souboru \texttt{presentation.tex}. Text uvnitř prostředí
\verb|\begin{frame}...\end{frame}| se vkládá jak do prezentace, tak do
článku, zatímco materiál mimo toto prostředí se nezobrazuje na snímcích, ale
zůstává součástí textové verze. Díky tomu může článek obsahovat podrobnější
vysvětlení nebo komentáře bez narušení struktury prezentace. Soubory
\texttt{preamble.tex} a \texttt{config.cfg} doplňují konfiguraci: první z nich
obsahuje balíčky a definice příkazů používané v prezentaci a sdílené v obou výstupech, zatímco druhý
slouží jako konfigurační soubor pro \TeX4ht a umožňuje upravit například \acro{CSS}
stylování webové verze nebo redefinice \LaTeX{}ových příkazů.


Obecně by mělo stačit upravovat obsah souboru \texttt{presentation.tex}, který 
obsahuje jak slidy pro Beamer, tak i poznámky pro článek. V souboru
\texttt{preamble.tex} můžeme přidat další balíčky nebo vlastní příkazy
a v \texttt{config.cfg} můžeme upravit styly webové verze článku. 

Obsah slidu s textovým popisem pro článek může vypadat například takto:

\begin{verbatim}
\begin{frame}[fragile]{Nadpis slidu}
\begin{itemize}
\item bod 1
\item bod 2
\end{itemize}
\end{frame}

Zde je textový popis mimo prostředí \texttt{frame}.
Zobrazí se pouze ve článku.
\end{verbatim}


Obsah uvnitř bloku \verb|\begin{frame}...\end{frame}| se zahrne
do prezentace i článku, ale následující odstavec -- mimo toto prostředí -- 
se zobrazí pouze v článku.

Pokud chceme obsah mimo prostředí \texttt{frame} zahrnout jak do prezentace, tak do článku,
můžeme použít příkaz \verb|\mode|:


\begin{verbatim}
\mode<beamer|article>{
\title{Strukturovaná šablona prezentace}
\author{Michal Hoftich}
\maketitle}
\end{verbatim}

Příkaz \verb|\|\verb+mode<beamer|article>{...}+ umožňuje zahrnout obsah, který se má
zobrazit jak v prezentaci, tak v handoutu. To se může hodit například pro titulní stranu
nebo obsah prezentace.


\begingroup
\sloppy
\printbibliography
\endgroup

\begin{summary}
The article presents a set of templates for the \TeX 4ht tool, which serves to convert
\LaTeX\ documents into \acro{HTML}. These templates greatly simplify the publication of various
types of documents on the web and bring modern capabilities for processing and automation.

The first template is intended for converting book-style documents into a web format.
It allows the text to be divided into individual chapters with automatically generated
navigation and support for responsive design, making the result easily readable even on
mobile devices.

The second template is used for creating statically generated blogs. Each post is written
as a separate \LaTeX\ document, which is converted to \acro{HTML} using \TeX 4ht. Subsequently,
these articles are processed by a static site generator such as Jekyll, which takes care
of assembling the entire blog, creating indexes, archives, and other navigation elements.

The third template focuses on converting presentations created in the Beamer environment
into so-called handouts -- overview materials for the audience. The result is a readable
and well-structured web document suitable for sharing after the lecture.

All templates are designed to work within GitHub Actions. This means that the documents
can be automatically compiled and published online whenever a change is made in the
repository. This approach ensures that the web version of the document is always up to date.

\keywords: \TeX 4ht, \LaTeX, \acro{HTML}, Jekyll, Beamer, GitHub Actions
\end{summary}
\end{document}
