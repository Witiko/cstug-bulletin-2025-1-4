\let\oldlooseness=\looseness
\RequirePackage{luatex85}
\PassOptionsToPackage{shorthands=off}{babel}
\makeatletter
\disable@package@load{fontenc}
\makeatother
\documentclass{csbulletin}
\selectlanguage{czech}
\usepackage[utf8]{inputenc}
\usepackage{luavlna, hologo}
\usepackage[strict]{lua-widow-control}
\usepackage{csquotes}
\usepackage[
  backend=biber,
  style=iso-numeric,
  sortlocale=cs,
  autolang=other,
  bibencoding=UTF8,
  mincitenames=2,
  maxcitenames=2,
  maxbibnames=3,
]{biblatex}
\addbibresource{uvodnik.bib}
\usepackage[
  implicit=false,
  hidelinks,
]{hyperref}

% Pomocná makra
\def\TUG{TUG}
\def\TUGboat{TUGboat}
\let\stress\emph
\let\makro\texttt
\makeatletter
\DeclareRobustCommand{\La}{L\kern-.36em%
        {\sbox\z@ T%
         \vbox to\ht\z@{\hbox{\check@mathfonts
                              \fontsize\sf@size\z@
                              \math@fontsfalse\selectfont
                              A}%
                        \vss}%
        }}
\makeatother
\def\AllTeX{(\La\kern-.075em)\kern-.075em\TeX}
\ExplSyntaxOn
\newcommand
  \acro
  [ 1 ]
  {
    \tl_set:Nn
      \l_tmpa_tl
      { #1 }
    \regex_replace_all:nnN
      { [^\d]+ }
      { \c{textsc} \cB\{ \c{MakeLowercase} \cB\{ \0 \cE\} \cE\} }
      \l_tmpa_tl
    \regex_replace_all:nnN
      { \d+ }
      { \c{oldstylenums} \cB\{ \0 \cE\} }
      \l_tmpa_tl
    \tl_use:N
      \l_tmpa_tl
  }
\ExplSyntaxOff

\begin{document}

% Metadata
\title{Úvodní slovo}
\EnglishTitle{Editorial}
\author{Vít Starý Novotný}
\podpis{Vít Starý Novotný, witiko@mail.muni.cz}
\maketitle

\vspace{-2em}

Vážení a milí příznivci a příznivkyně \TeX u!

\medskip

Čtyřčíslo \emph{Zpravodaje}, které držíte v rukou, přináší řadu podnětných článků:

\begin{itemize}
\item \emph{Zpráva z konference \acro{TUG 2025}} od autora tohoto textu zachycuje jeho účast na letošní konferenci \acro{TUG}, která se po čtrnácti letech vrátila do indické Kéraly. Článek detailně popisuje průběh celého setkání, včetně poznámek z valného shromáždění \acro{TUG}u, které proběhlo v rámci konference.

\item \emph{Publikace \LaTeX ových dokumentů na webu pomocí \TeX4ht a GitHub Actions} od Michala Hofticha rozvíjí jeho příspěvek~\cite{ossconf} z letošní konference OSSConf pořádané tradičně na Žilinské univerzitě v Žilině. Autor v něm představuje tři šablony pro nástroj \TeX4ht, který s jejich pomocí převádí \LaTeX ové dokumenty do formátu \acro{HTML} určeného pro webové publikace.

\item \emph{Co se děje ve světě \ConTeXt u}, \emph{Vybrané \TeX ovské balíčky z rodiny games} a \emph{Sadzba zdrojového kódu v \LaTeX u pomocou balíka minted} navazují na tradici referátů~\cite{finalwork} studujících předmětu Elektronická příprava dokumentů, který na Fakultě informatiky Masarykovy univerzity již čtyřiadvacet let vyučuje předseda sdružení Petr Sojka.

V prvním článku Jana Slámová představuje některé novinky z formátu \Hologo{ConTeXt}.
Ve druhém Branislav Hitzinger přibližuje vybrané balíčky pro sazbu herních plánů deskových her~\cite{games}.
Ve třetím Filip Blaho seznamuje začátečníky s \LaTeX ovým balíčkem minted pro sazbu zdrojového kódu a pokročilé čtenáře seznamuje s novinkami ve třetí verzi balíčku, která vyšla loni v září.
\oldlooseness=-1
\end{itemize}

Věřím, že Vás obsah čísla potěší a že nám i nadále zachováte svou přízeň.

\vspace{-1.2em}
\section{Odkazy}
\vspace{-0.4em}
\printbibliography[heading=none]

\makeatletter
\csbul@podepis
\makeatother

\medskip

P. S.: V tiráži si můžete všimnout, že redakce \emph{Zpravodaje} doznala od předešlého čísla významných personálních změn. Podrobněji se k nim vrátím v příštím čísle po valném shromáždění spolku, které by mělo nové složení redakce schválit.
\end{document}
